% \documentclass[conference]{IEEEtran}
% \IEEEoverridecommandlockouts
\documentclass[sigconf,review,anonymous]{acmart}
\renewcommand\footnotetextcopyrightpermission[1]{}
% The preceding line is only needed to identify funding in the first footnote. If that is unneeded, please comment it out.
% \usepackage{cite}

\settopmatter{printfolios=true}
% make references clickable 
\usepackage[]{hyperref}

\settopmatter{printfolios=true}
\settopmatter{printacmref=false}
\pagestyle{plain}
\usepackage{tikz}
\usepackage{float}
\usepackage{amsmath}
\usepackage{xspace}
\usepackage{cleveref}
\usepackage{balance}
\usepackage{url}
\usepackage{siunitx}
\usepackage{comment}
\usepackage{xcolor}
\usepackage{enumitem}
\usepackage{listings}
\usepackage{makecell}
\usepackage{algorithm2e}
\usepackage{multirow}
% \usepackage{minted}
\usepackage{graphicx}
\usepackage[thicklines]{cancel}
\usepackage{caption}
% \def\BibTeX{{\rm B\kern-.05em{\sc i\kern-.025em b}\kern-.08em
    % T\kern-.1667em\lower.7ex\hbox{E}\kern-.125emX}}

\author{Yusheng Zheng}
\affiliation{%
  \institution{Eunomia. Inc.}
  \country{USA}
}
\email{yunwei356@gmail.com}
\author{Yiwei Yang}
\affiliation{%
  \institution{UC Santa Cruz}
  \country{USA}
}
\email{yyang363@ucsc.edu}

\author{Haoqin Tu}
\affiliation{%
  \institution{UC Santa Cruz}
  \country{USA}
}
\email{tuisaac163@gmail.com}

\author{Yuxi Huang}
\affiliation{% 
  \institution{Eunomia. Inc.}
  \country{USA}
}
\email{yuxi4096@gmail.com}

% \title{Do we really know how the system works? Automatic agent system for code behavior analysis}
\title{Code-Survey: An LLM-Driven Methodology for Analyzing Large-Scale Codebases}

\newcommand{\hq}[1]{\textcolor{blue}{[Haoqin: #1]}}

\begin{document}

% \thanks{Identify applicable funding agency here. If none, delete this.}


% \author{

% \IEEEauthorblockN{1\textsuperscript{st} Yusheng Zheng}
% \IEEEauthorblockA{\textit{Eunomia. Inc.} \\
% % \textit{name of organization (of Aff.)}\\
% City, Country \\
% email address or ORCID}
% \and

% \IEEEauthorblockN{2\textsuperscript{nd} Yiwei Yang}
% \IEEEauthorblockA{\textit{UC Santa Cruz} \\
% % \textit{name of organization (of Aff.)}\\
% City, Country \\
% email address or ORCID}
% }
\begin{abstract} 

Modern software systems like the Linux kernel are among the world's largest and most intricate codebases, continually evolving with new features and increasing complexity. Understanding these systems poses significant challenges due to their scale and the unstructured nature of development artifacts such as commits and mailing list discussions. We introduce Code-Survey, the first LLM-driven methodology designed to systematically explore and analyze large-scale codebases. The central principle behind Code-Survey is to treat LLMs as human participants, acknowledging that software development is also a social activity and thereby enabling the application of established social science techniques. By carefully designing surveys, Code-Survey transforms unstructured data—such as commits, emails—into organized, structured, and analyzable datasets. This enables quantitative analysis of complex software evolution and uncovers valuable insights related to design, implementation, maintenance, reliability, and security.

To demonstrate the effectiveness of Code-Survey, we apply it to the Linux kernel's eBPF subsystem. We construct the Linux-bpf dataset, comprising over 670 features and 16,000 commits from the Linux community. Our quantitative analysis uncovers important insights into the evolution of eBPF, such as development patterns, feature interdependencies, and areas requiring attention for reliability and security—insights that have been initially validated by eBPF experts. Furthermore, Code-Survey can be directly applied to other subsystems within Linux and to other large-scale software projects. By providing a versatile tool for systematic analysis, Code-Survey facilitates a deeper understanding of complex software systems, enabling improvements across a variety of domains and supporting a wide range of empirical studies. The code and dataset is open-sourced in https://github.com/eunomia-bpf/code-survey
\end{abstract}

% \begin{abstract}
% Researchers often refer to the Linux kernel as a background system, but do they truly understand how one of the world's largest systems is designed?\hq{what is `a background system'? I think we can just rephrase this sentence in a declarative way: Linux kernel is a fundamental component in computer science, but researchers fail to fully comprehend such complex system in design.}
% Unlike research projects with clear design goals, real-world systems evolves as new features are continuously added over time. For instance, eBPF in Linux has gained significant attention in both industry and the community. While bpf\_link has been in the kernel for four years and plays a key role in connecting different runtime components to reduce complexity—supporting use cases like observability, security, networking, drivers, and even potential disk I/O and memory management—it remains rarely discussed outside of Linux source code.

% With the rise of AI and LLM Agents, we propose the first approach to help researchers better understand real-world systems and uncover insights related to design, implementation, maintenance, reliability and security inside them. 
% % Code-survey is the first step toward bridging the gap between design, implementation, maintenance, reliability and security in one of the largest system in the world. 
% We introduce Code-survey, an automatic framework designed for transforming unstructured Linux kernel commits and emails into structured data, which enables human experts and LLM agents to efficiently query and analyze it.
% To further verify the effectiveness of our Code-survey, we leverage the technique and construct the Linux-bpf dataset. Our new data consists of more than 670 features, 12k commits, and 150k mails in the Linux community.
% Instead of generating code, answering documents or static analysis, the results reveal important insights about the eBPF subsystem\hq{What is the insight?}, confirmed by experts\hq{In which aspect?}. 
% Code-survey can also be applied to other subsystems\hq{For example?} directly.

% % We introduce Code-survey and the Linux-bpf dataset, which consist of 670+ features, 12k+ commits, 150k+ mails. Instead of generating code, answering documents or static analysis, Code-survey transforms unstructured Linux kernel commits and mails into structured data, enabling human experts or llm agents to query and analyze it. The results reveal important insights about the eBPF subsystem, confirmed by experts. Code-survey can also be applied to other subsystems directly.

% \end{abstract}


\maketitle

\section{Introduction}

Software systems are increasingly complex, with real-world applications often requiring continuous development over time. Unlike research projects with clearly defined goals and controlled development processes, real-world systems evolve organically. Features are added incrementally, bugs are fixed, and design decisions may be modified years after initial implementation. This complexity is further exacerbated by the need to balance backward compatibility, feature requests, performance improvements, and security patches. As systems evolve, tracing the original intent behind design decisions or understanding the rationale for modifications becomes difficult. This lack of transparency often leads to technical debt, regressions, and challenges in maintaining system reliability.

One of the most prominent examples of a continuously evolving real-world system is the Linux kernel. Serving as the backbone of countless devices and platforms—from cloud servers to mobile devices—the Linux kernel must support a wide array of features while maintaining rigorous performance standards. The \textit{extended Berkeley Packet Filter} (eBPF)\cite{ebpf} subsystem exemplifies this complexity, as it supports critical functionalities such as observability\cite{shen2023network}, networking\cite{vieira2020fast}, and security\cite{deri2019combining}. Despite its significance, much of the development history and design rationale behind eBPF remains underexplored. For example, features like \texttt{bpf\_link}\cite{bpflink}, which provide a new abstraction for attaching programs to events, have been part of the Linux source code for several years but have received little attention outside of kernel developers. Similarly, the increasing use of kfuncs \cite{kfuncs} as replacements for helpers, the growing complexity of control-plane applications, and efforts toward making eBPF Turing complete have not been extensively explored in the broader community.

Understanding the evolution of features in large, complex codebases is a significant challenge in software development\cite{godfrey2008past,mens2008introduction}. Traditional methods, such as static analysis and manual code review, are limited in capturing the full context of a system's growth and change, and require substantial human effort. Unstructured data sources, like commit messages and mailing list discussions, contain valuable insights but are difficult to analyze systematically. Consequently, important information about design decisions, feature evolution, and system behavior is often hidden within large volumes of unstructured text. This makes it nearly impossible to answer questions such as: ``Why was this feature added?'', ``How has this feature evolved?'', or ``What were the discussions that led to this change?''

Recent advancements in artificial intelligence, particularly in Large Language Models (LLMs) like GPT-4o\cite{gpt4o} and O1~\cite{o1}, have opened new opportunities to address these challenges. LLMs have shown great promise in automating software engineering tasks such as code generation\cite{zheng2024kgent}, bug detection\cite{li2024enhancing}, debugging\cite{chen2023teaching}, and error fixing\cite{deligiannis2023fixing}. However, most current applications of LLMs focus on well-defined tasks involving source code or documented APIs. Little work has explored how LLMs can be applied to understand the long-term evolution of large-scale, real-world software systems.

In this paper, we introduce \emph{Code-survey}, a novel methodology that leverages Large Language Models (LLMs) to systematically transform unstructured data—such as commit histories and emails\cite{linux,tan2019communicate,schneider2016differentiating}—into structured datasets for large-scale software analysis. Drawing inspiration from sociological surveys that utilize human participants to gather extensive data, Code-survey employs LLMs to emulate this process, enabling efficient and scalable analysis of software development artifacts. By focusing on the vast amount of text produced during software development, Code-survey allows us to answer questions that were previously difficult to tackle using only structured data in large real-world systems. Through data analysis enabled by Code-survey, we can explore questions such as:


\begin{itemize}
    \item ``How do new feature introductions impact the stability and performance of existing kernel components?''
    \item ``Are there identifiable phases in the lifecycle of a feature, such as initial development, stabilization, and optimization?''
    \item ``How has the functionality of a specific eBPF feature evolved over successive commits?''
    \item ``Which components or files in the Linux kernel have the highest bug frequency?''
    \item ``What lessons can be learned from the development history of kernel eBPF that can be applied to improving other eBPF runtimes?''
    \item ``What dependencies have emerged between features and component?''
\end{itemize}

To demonstrate the efficacy of Code-survey, we apply it to the \textit{Linux-bpf dataset}, which contains over 670 features, 15,000 commits, and 150,000 emails related to the development of the eBPF subsystem. Through this structured dataset, we uncover new insights into the design and evolution of features like \texttt{bpf\_link}, and highlight trends that were previously hidden within the unstructured data. These insights have been initially confirmed by eBPF experts.

The key contributions of this paper are as follows:

\begin{itemize}
    \item We introduce \emph{Code-survey}, a novel methodology that leverages LLMs to transform unstructured data produced in software development into structured datasets via surveys, enabling systematic analysis of software evolution. To the best of our knowledge, \emph{Code-survey} is the first methodology that leverages LLMs for the systematic analysis of large-scale codebases.
    \item We present the \textit{Linux-bpf dataset}, a structured dataset comprising over 670 features, 15,000 commits, and 150,000 emails related to the eBPF subsystem in the Linux kernel.
    \item We apply the \emph{Code-survey} methodology to build an LLM-driven agent system, allowing us to perform systematic analysis on the \textit{Linux-bpf dataset}.
    \item We demonstrate that the \emph{Code-survey} methodology reveals new insights into the evolution of eBPF kernel features that are impossible to uncover using traditional methods. By combining traditional data analysis methods with eBPF experts' domain knowledge and historical context, we also initially verified the consistency and correctness of the data.
    \item We identify and highlight under-explored areas of eBPF to support various use cases with the help of \emph{Code-survey}, pointing out interesting research directions.
\end{itemize}

The remainder of this paper is structured as follows. We review background in Section~\ref{sec:related}, followed by a detailed explanation of the Code-survey methodology in Section~\ref{sec:methodology}. Section~\ref{sec:analysis} presents our analysis of the Linux-bpf dataset and the insights gained from it. Sections~\ref{sec:limitations} and~\ref{sec:future} conclude with a discussion of current limitations and future work. All artifacts are open-sourced at \url{https://github.com/eunomia-bpf/code-survey}.

\section{Background}
\label{sec:related}

This section discuss the role of Large Language Models in software development, the complexities of Linux kernel development, and the importance of survey methodologies in empirical software engineering research.

\subsection{LLMs in Software Development}

Large Language Models (LLMs), such as GPT-4o and Claude, have significantly impacted software development~\cite{jin2024llms}, particularly in automating tasks like code generation, debugging, and testing. Tools like GitHub Copilot~\cite{copilot} leverage these models to enhance developer productivity by providing intelligent code suggestions. Despite these advancements, challenges such as hallucinations—where incorrect but plausible code is generated—still persist~\cite{fan2023large,ji2023survey}. Moreover, most research focuses on well-defined tasks, and exploring the design and evolution of large-scale software systems using LLMs remains underexplored.

Additionally, using LLMs for summarizing test results, decision-making, and converting unstructured data into structured formats is becoming increasingly common in both academia~\cite{jin2024comprehensive,iourovitski2024grade,patel2024lotus} and industry~\cite{llmnvida}. This capability is especially valuable in environments where massive amounts of unstructured data—such as logs, emails, or messages—exist. The ability to systematically extract insights from this data enables more efficient analysis and has been widely applied to tasks like market research~\cite{brand2023using}.

\subsection{Software Evolution and Its Challenges}
Software evolution involves the continuous modification and adaptation of systems to meet changing requirements~\cite{lehman1996laws}. According to Lehman's laws, systems must evolve to remain useful, but this often increases complexity without proactive management. In large-scale systems like the Linux kernel~\cite{linux}, evolution is non-linear, involving numerous contributors and revisions, which leads to intricate interdependencies~\cite{israeli2010linux}. The Linux kernel generates vast unstructured data, including commit logs and email threads, which traditional analysis methods struggle to process~\cite{mens2008introduction}. These artifacts contain rich contextual information about design decisions, but their volume and unstructured format hinder conventional techniques. Additionally, evolving systems accumulate \emph{technical debt}~\cite{brown2010managing}, increasing maintenance costs and reducing reliability. Addressing these challenges requires innovative approaches capable of handling large-scale unstructured data to provide actionable insights.



\subsection{Survey Methodology and Empirical Studies}

Empirical studies in software engineering~\cite{perry2000empirical} are crucial for understanding how software evolves and how development practices affect system reliability, performance, and maintainability. Traditional surveys rely on structured questionnaires or interviews but are limited by scale and biases such as subjective recall or incomplete responses. In contrast, the \textit{Code-Survey} methodology automates data collection using LLMs to extract structured insights from unstructured data like commit histories and mailing lists. This approach enables large-scale analysis and allows us to answer questions that traditional methods cannot.


\section{\sys: A Multi-Agent Framework for Socio-Technical Analysis}

To systematically analyze the decades of unstructured development history within the Linux kernel, we designed \sys, a multi-agent framework. Its core principle is to create a high-fidelity simulation of the collaborative, role-based social structure of the open-source community itself. Instead of relying on a single monolithic LLM or generic roles, \sys instantiates \textbf{Empirically-Grounded Personas}: agents representing actual key developers. This allows us to capture the nuanced and often competing perspectives that shape the kernel's evolution.

This socio-technical approach allows us to transform unstructured artifacts into structured, analyzable data by capturing not just \emph{what} changed, but the diverse perspectives of the specific individuals who drove those changes.

\subsection{Design Goals}

The design of \sys is guided by four core principles derived from observing the Linux development process.

\begin{table}[h]
\centering
\begin{tabular}{p{1cm}|p{4cm}|p{9cm}}
\hline
ID & Requirement & Rationale \& Mapping to Linux Practice \\
\hline
\textbf{G-1} & \textbf{Faithful Stakeholder Modeling} & Agents must embody the distinct perspectives of \textbf{real kernel participants}. This mirrors the real-world tension between goals like feature velocity, long-term stability (a maintainer's view), and risk mitigation (a security developer's view). \\
\textbf{G-2} & \textbf{Question-Driven, Evidence-Based Analytics} & The system must answer high-level natural language questions with quantitative analysis and \textbf{explicit, traceable evidence}. This reflects the kernel community's norm where any claim must be backed by references to specific commits, mailing list threads, or performance data. \\
\textbf{G-3} & \textbf{Scalable and Economical Operation} & The framework must process millions of artifacts cost-effectively. We implement a \textbf{tiered inference pipeline}, where cheap, heuristic-based agents handle the majority of simple tasks, reserving powerful models for the complex reasoning performed by the core developer agents. This mimics the kernel's use of automated bots to triage issues before they reach human maintainers. \\
\textbf{G-4} & \textbf{Measurable Reliability and Consensus} & The system must quantify its own internal agreement and reliability. By measuring inter-agent agreement, we build a trust model analogous to the kernel's \texttt{Reviewed-by} and \texttt{Tested-by} tags, ensuring that outputs are the result of a \textbf{consensus between key developer perspectives}. \\
\hline
\end{tabular}
\end{table}


\subsection{Simulating the Kernel Development Workflow}

The \sys architecture answers a user's query by simulating the lifecycle of a change within the Linux kernel community. The process is broken into phases that mirror how a patch moves from submission to integration.

\emph{Figure X: The architecture of \sys, which simulates the kernel development workflow to analyze historical artifacts.}

\begin{enumerate}
\item \textbf{Phase 1: Query Decomposition \& Artifact Retrieval.} The process begins when the \textbf{Orchestrator} agent parses a user's natural language query. It identifies the relevant historical artifacts—a set of commits, patch series, and mailing list threads—that serve as the subject of the analysis. This is analogous to a developer focusing on a specific problem or proposed change.

\item \textbf{Phase 2: Automated Triage (CI Pipeline Simulation).} The retrieved artifacts are first passed to a \textbf{Triage Crew} of lightweight, heuristic-based agents. This simulates the kernel's automated CI systems and bots. \texttt{SyzbotTwin} flags artifacts related to fuzzer-found bugs, while \texttt{CheckpatchBot} annotates for style or common errors. This phase enriches the artifacts with objective metadata before they undergo expert human review.

\item \textbf{Phase 3: Asynchronous Community Review (Mailing List Simulation).} This is the core analytical phase, simulating the peer review process on a mailing list. A dynamically assembled \textbf{Persona Crew}, consisting of Empirically-Grounded Personas for relevant developers (e.g., Alexei Starovoitov, Brendan Gregg), analyzes the artifacts in parallel. Each agent assesses the change from its unique perspective, conditioned by its personal knowledge base, and records its findings in a structured \textbf{survey}.

\item \textbf{Phase 4: Consensus Building \& Maintainer Decision (Resolution Simulation).} This phase simulates the process of achieving community consensus and receiving a maintainer's final judgment. The \textbf{Critic-Resolver} agent aggregates the surveys from the Persona Crew and calculates an agreement score (e.g., Cohen's Kappa, $\kappa$). If disagreement is high, a "debate" is triggered where agents must argue their positions by citing evidence, mimicking a mailing list discussion. The final, resolved view represents the community consensus, analogous to a maintainer applying a \texttt{Reviewed-by} tag or merging the patch.

\item \textbf{Phase 5: Insight Synthesis \& Reporting.} In the final phase, the \textbf{Data-Synthesizer} agent takes the structured, consensus-driven data from the simulation. It performs any required quantitative analysis (e.g., running \texttt{DuckDB} queries), generates charts and tables, and composes a final, human-readable report that answers the user's original query.
\end{enumerate}


\subsection{Empirically-Grounded Personas}

The key to \sys's fidelity is its use of Empirically-Grounded Personas. These are agents representing real community members, with each agent's behavior and analytical lens shaped by that person's actual history of contributions.

\textbf{Each persona is defined by a unique Knowledge Base}, a personal memory built from that individual's digital footprint. This knowledge base is implemented as a dedicated vector store for Retrieval-Augmented Generation (RAG) and contains:

\begin{itemize}
\item All \textbf{commits authored} by the developer.
\item All \textbf{mailing list emails} sent by them.
\item All patches they have formally reviewed (via \texttt{Reviewed-by:} tags).
\item Their documented positions in design discussions and talks.
\end{itemize}

This allows for highly specialized analysis. For example:

\begin{itemize}
\item An \textbf{Alexei Starovoitov agent} (eBPF maintainer) analyzes a change with a primary focus on verifier safety and API stability, drawing on its memory of thousands of accepted or rejected BPF patches.
\item A \textbf{Brendan Gregg agent} (observability expert) assesses the same change based on its utility for performance tracing tools, referencing its knowledge base of past \texttt{perf} and \texttt{bpftrace} developments.
\item A \textbf{KP Singh agent} (security developer) evaluates it from a security perspective, comparing code patterns to its memory of past discussions on Linux Security Modules (LSMs) and BPF hardening.
\end{itemize}

This method transforms the implicit, individual expertise buried in project history into an explicit, structured format ready for quantitative analysis and deep socio-technical insight.

Of course. Here is the "Illustrative Workflow in Action" section, which details how the \sys framework uses its survey-based, multi-agent process to answer a complex question.


\subsection{Illustrative Workflow in Action}

To demonstrate the framework's capability, consider how \sys would answer the query: \textbf{"How did the refactoring of verifier loops after kernel v5.10 affect BPF program size limits?"}

The framework follows a story-like progression that simulates the community's problem-solving process.

\begin{enumerate}
\item \textbf{Survey Design \& Scoping}: The \textbf{Orchestrator} agent first parses the query. It identifies the core topics (verifier, performance, limits) and the relevant timeframe (post-v5.10). Based on this, it selects a tailored survey schema designed for analyzing architectural changes. This survey includes questions about motivation, technical risk, performance impact, and security implications, capturing multiple stakeholder perspectives. It then retrieves the relevant commits and mailing list discussions for analysis.

\item \textbf{Multi-Agent Analysis}: The artifacts are passed to the \textbf{Persona Crew}. For this query, agents for \textbf{Alexei Starovoitov}, \textbf{Daniel Borkmann}, and \textbf{Brendan Gregg} are activated. Each agent independently analyzes the artifacts and completes the survey from its unique perspective:
    \begin{itemize}
    \item The \textbf{Alexei Starovoitov agent}, drawing from its maintainer memory, completes its survey noting the motivation was "refactoring for performance" with a "medium" technical risk.
    \item The \textbf{Daniel Borkmann agent} concurs on the motivation but assesses the risk as "low," citing its knowledge of the extensive test suite that was developed concurrently.
    \item The \textbf{Brendan Gregg agent}, focusing on observability, highlights in its survey that the primary "impact" was enabling more powerful tracing tools.
    \end{itemize}

\item \textbf{Consensus Building}: The \textbf{Critic-Resolver} agent gathers the completed surveys. It calculates the inter-agent agreement (Cohen's kappa, $\kappa$) and flags a disagreement on the "risk" assessment ("medium" vs. "low"). This triggers a simulated debate where the agents exchange their core evidence. The Alexei agent cites a past verifier bug, while the Daniel agent provides a link to the specific test-suite emails. They reconcile to a final consensus assessment of "low-to-medium risk, mitigated by extensive testing," which is recorded in the final survey data.

\item \textbf{Data Integration \& Synthesis}: Finally, the \textbf{Data-Synthesizer} agent takes the final, structured survey data. It performs a quantitative query on its dataset to plot the \texttt{BPF\_MAX\_INSNS} constant over different kernel versions, generating a chart that visually confirms the impact on program size limits. This transforms the multi-perspective analysis into a queryable insight, effectively integrating 30+ years of unstructured knowledge. The agent then composes the final answer, weaving together the narrative (the "why" from the surveys) and the quantitative proof (the chart) into a single, evidence-backed report.
\end{enumerate}


\section{Case Study: eBPF}
\label{sec:analysis}

% In this section, we apply the \emph{\sys} methodology to analyze the evolution and development of the eBPF subsystem in the Linux kernel. Our goal is to uncover insights into the lifecycle, stability, and design decisions of key eBPF features. The survey results are validated through expert review and both quantitative and qualitative analyses.
In this section, we apply the \sys multi-agent framework to analyze the evolution and development of the eBPF subsystem in the Linux kernel. Our goal is to uncover how different stakeholders perceive the lifecycle, stability, and design decisions of key eBPF features. The multi-agent analysis reveals divergent perspectives that mirror real community dynamics, validated through expert review and consensus metrics.

The Extended Berkeley Packet Filter (eBPF)~\cite{ebpf} is a rapidly evolving subsystem in the Linux kernel that allows users to run sandboxed programs in kernel space without modifying the kernel itself~\cite{lim2024safebpf}. Originally developed for packet filtering, eBPF now supports diverse use cases such as performance tracing, security monitoring, and system observability, and has been expanded to multiple platforms~\cite{windows-ebpf,zheng2023bpftime}. The academic community has identified current problems and limitations of eBPF, proposing several works to improve aspects like the verifier and deployment.

\subsection{Research Questions for Survey Evaluation}

% To ensure that the LLM does not answer questions randomly, we evaluate the effectiveness and correctness of the \emph{\sys} results by exploring the following high-level research questions:
To ensure that our multi-agent system provides meaningful insights rather than random responses, we evaluate the effectiveness and correctness of the \sys results by exploring the following high-level research questions:

\begin{itemize}
    \item \textbf{Correctness of Survey Responses:} How can we ensure that survey responses reflect accurate and relevant information about the system's features and commits?
    \item \textbf{Consistency Across Similar Questions:} Are similar questions answered consistently across different but related features or subsystems?
    \item \textbf{Coverage of Survey Questions:} Do the survey questions comprehensively cover all relevant aspects of the feature or subsystem under analysis?
    \item \textbf{Insight from Survey:} Can the survey data help users analyze the design, implementation, maintenance, reliability, and security evolution, and gain valuable insights?
    \item \textbf{Clarity and Ambiguity in Responses:} Are the survey responses clear and unambiguous, making them actionable for further analysis?

    \item \textbf{Expert Confirmation:} How do experts rate the accuracy of the survey's generated insights?
\end{itemize}

\subsection{Survey Design and Implementation}

To gain deeper insights into the design and evolution of the eBPF subsystem, we developed a comprehensive survey aimed at classifying commits within the Linux eBPF subsystem. The survey evaluates specific aspects of each commit by analyzing commit messages and associated code changes, with questions covering commit classification, complexity, affected components, and use cases.

\subsection{Implementing the Survey Using LLM Agents}

To efficiently process and analyze the vast number of commits in the Linux eBPF subsystem, we leveraged LLMs to automate our survey. We developed Assistant Agents using GPTs~\cite{gpts} for generating survey responses and assisting in analyzing results. By utilizing the GPT-4o~\cite{gpt4o} LLM model, we transformed unstructured commit data into structured responses aligned with our survey definitions. We initially applied this method to over 15,000 commits spanning eight years and plan to expand it to include emails and patches.

\subsubsection{Commit Survey Methodology}

The AI analyzes commits through: (1) data extraction of commit details, (2) prompt construction with survey questions, (3) analysis of messages and code changes, (4) feedback loop for incomplete responses, and (5) code generation for quantitative analysis.

We used the GPT-4o model for its strong language understanding and ability to handle technical content, making it well-suited for analyzing kernel commits.

\subsubsection{Enhancing Survey Accuracy}

Accuracy can be enhanced through: multiple survey runs (we ran once due to budget, achieving <1\% error), domain-specific fine-tuning or advanced models like O1~\cite{o1}, clearer prompts, and multi-step processes for complex commits.

LLM automation enabled efficient processing of tens of thousands of commits, transforming unstructured data into structured insights.

% \subsection{The Commit Dataset}

% The \texttt{commit\_survey.csv} dataset provides metadata for over 15,000 Linux kernel commits, including commit types, messages, timestamps, and affected components. It categorizes and classifies commits, focusing on the eBPF subsystem.

% \subsubsection{Dataset Overview}

% The dataset contains the following fields:

% \begin{itemize}
%     \item \textbf{Commit Metadata}: Unique commit IDs, author and committer details, and timestamps.
%     \item \textbf{Commit Messages and File Changes}: Descriptions of the changes in each commit.
%     \item \textbf{Classification}: Types such as bug fixes, feature additions, or merges.
%     \item \textbf{Complexity}: Based on the number of files and lines changed.
%     \item \textbf{Components}: Affected implementation and logic components.
%     \item \textbf{Use Cases}: Related subsystems and modules.
% \end{itemize}

% \subsubsection{Key Findings from Dataset Analysis}

% Our analysis reveals several important patterns in eBPF subsystem development:

% \textbf{Commit Classification:} Most commits focus on bug fixes and code cleanups, reflecting efforts to maintain code quality. Significant attention goes to testing infrastructure changes, emphasizing robustness. New features constitute a considerable portion of commits, while merge commits are commonplace in Linux kernel development.

% \textbf{Commit Complexity:} Most commits are simple, involving small changes, while complex changes constitute a smaller portion. This distribution suggests eBPF development follows an incremental improvement pattern.

% \textbf{Implementation Components:} Test cases and build scripts are significantly affected, highlighting continuous testing and build improvements. The \texttt{libbpf} library is a key component in the kernel eBPF toolchain, receiving considerable attention. Substantial development occurs in other kernel subsystems, particularly eBPF events. Frequent updates to the verifier and helpers indicate efforts to enhance functionality and ensure program safety. Some commits appear unrelated to the eBPF subsystem.

% \textbf{Logic Components:} General utilities, such as tools and scripts, receive the most updates, followed by runtime features like helpers and kernel functions, which are consistently enhanced. eBPF event logic and instruction handling are also frequently updated to ensure robustness and functionality.

% \textbf{Use Cases and Events:} While most commits enhance the core eBPF infrastructure—including the verifier and runtime components—significant development also extends to networking-related features such as socket and XDP programs, which receive substantial attention. Additionally, tracing tools like tracepoints and kprobes highlight eBPF's crucial role in system diagnostics and debugging.


\subsection{Development Artifacts: Commits and Communications}

Our analysis encompasses over 15,000 Linux kernel commits and 150,000+ mailing list emails from the eBPF subsystem, captured in the \texttt{commit\_survey.csv} dataset. This comprehensive dataset includes commit metadata (IDs, authors, timestamps), commit messages and file changes, classification categories (bug fixes, features, merges), complexity metrics based on files and lines changed, affected components (implementation and logic), and related use cases and subsystems. The email corpus provides additional context about design decisions, community discussions, and feature evolution that complements the commit data.

Key findings reveal distinct development patterns: Most commits focus on bug fixes and code cleanups, with significant attention to testing infrastructure, reflecting the community's emphasis on stability and robustness. The complexity distribution shows predominantly simple, incremental changes, characteristic of mature kernel development. Implementation analysis highlights that test cases and build scripts receive the most updates, while the \texttt{libbpf} library receives considerable attention. Frequent updates to the verifier and helpers indicate efforts to enhance functionality and ensure program safety. Notably, some commits appear unrelated to the eBPF subsystem due to broad filtering criteria. For logic components, general utilities receive the most updates, followed by runtime features like helpers and kernel functions. Networking features (socket, XDP) and tracing tools (kprobes, tracepoints) represent major development areas, underscoring eBPF's dual role in both network processing and system observability. The email discussions reveal heated debates around verifier complexity, API stability, and the balance between feature velocity and kernel maintainability—tensions that our multi-agent framework captures through divergent agent perspectives.

\subsection{Correctness of Survey Responses}

To ensure that the survey responses accurately reflect system features and commits, we validated the results by randomly sampling the data and cross-referencing with expert knowledge and processing logs. Specifically, we addressed two common issues: the handling of merge commits and the identification of commits unrelated to the eBPF subsystem.

We observed discrepancies in how merge commits were classified between the commit classification and the major implementation component perspectives. To address this, we analyzed the top merge commit messages and found that some merge commits were categorized based on their predominant effect on a specific component rather than being uniformly labeled as merge commits. By clarifying this distinction in the survey questions, we ensured that the classification system accurately reflected the commit's impact across different perspectives.

\textbf{Example:}
\begin{verbatim}
Top Commit (Classification but not Implementation):
3    Merge branch 'bpf-fix-incorrect-name-check-pass'
25   Merge branch 'vsc73xx-fix-mdio-and-phy' Pawel...
\end{verbatim}

\subsubsection{Non-Related eBPF Subsystem Commits}

We noted that commits not directly related to the eBPF subsystem were sometimes included due to broad filtering criteria (e.g., \texttt{--grep=bpf}). Upon reviewing these commits, we confirmed that some mentioned ``bpf'' but addressed unrelated or peripheral issues.

\textbf{Example:}
\begin{verbatim}
Sample 'Not related to eBPF' Commit Messages:
17    bonding: fix xfrm real_dev null pointer...
21    btrfs: fix invalid mapping of extent xarray st...
\end{verbatim}

\subsubsection{Consistency Across Similar Questions}

We checked for consistency in responses by comparing related questions, focusing on the number of merge commits in commit classifications and complexities, as well as the number of commits unrelated to the eBPF subsystem in the implementation and logic components.

\textbf{Example 1: Merge Commits}
\begin{itemize}
    \item \textbf{Commit Classification:} ``It's like a merge commit.'' (2,130 responses)
    \item \textbf{Commit Complexity:} ``Merge-like. The commit merges branches.'' (2,132 responses)
\end{itemize}

\textbf{Example 2: Unrelated Components}
\begin{itemize}
    \item \textbf{Implementation Component:} ``Not related to eBPF.'' (773 responses)
    \item \textbf{Logic Component:} ``Not related to eBPF.'' (766 responses)
\end{itemize}

The close alignment of these numbers demonstrates consistent identification of unrelated components. With a low misclassification rate (under 0.05\% for total commits), our data shows high consistency, supporting the reliability of the survey design.

\subsection{Timeline Analysis of Commits}

Analyzing the timeline of commits provides valuable insights into the evolution of the eBPF subsystem over time. By visualizing the distribution of commit types, complexities, and major components across different periods, we can identify trends and patterns in the development of eBPF features.

The data was processed by cleaning to remove irrelevant commits, smoothing using a 3-month average to reduce noise and highlight long-term trends, and treating single-component \texttt{Merge} commits as regular commits while removing multi-component \texttt{Merge} commits, such as mainline merges. The time span covers from 2017 to the end of 2024, encompassing over 15,000 commits.

\subsubsection{Commit Classification Over Time}

\begin{figure}[ht]
    \centering
    \includegraphics[width=\linewidth]{feature-analysis/timeline_commit_classification_smoothed.png}
    \caption{Commit Classification Over Time}
    \label{fig:timeline_commit_classification_smoothed}
\end{figure}

Figure~\ref{fig:timeline_commit_classification_smoothed} shows the distribution of different types of commits over time.

The development began to grow significantly in 2017, with limited addition of test cases initially, which continued to improve over time. New feature development follows a cyclical pattern, with notable spikes around 2020 and 2021. After 2021, the number of cleanups and refactorings increases significantly, while new feature additions decline, indicating a shift in focus towards code maintainability and stability. A decline in cleanup commits after 2023 suggests that while new features continue to be added, the emphasis has shifted more towards stabilization and optimization.

\subsubsection{Commits Related to Major Implementation Components Over Time}

\begin{figure}[ht]
    \centering
    \includegraphics[width=\linewidth]{feature-analysis/timeline_major_related_implementation_component_smoothed.png}
    \caption{Commits Related to Major Implementation Components Over Time}
    \label{fig:timeline_major_related_implementation_component_smoothed}
\end{figure}

Figure~\ref{fig:timeline_major_related_implementation_component_smoothed} illustrates the evolution of major implementation components in the Linux eBPF subsystem.

Most components experienced their highest activity between 2017 and 2022, reflecting the rapid development of eBPF features during this period. The \texttt{libbpf} library saw the most dramatic increase, while the JIT compiler was most frequently updated around 2018. The rise in test cases also reflects the growing importance of a robust testing framework in this field.

After peaking around 2021--2022, several components show a decline or stabilization in activity. This indicates that many eBPF components have entered a phase of optimization and maintenance rather than new feature development, while testing continues to increase coverage. The verifier shows modest activity throughout the observed period but seems to increase in 2023--2024, which may reflect renewed research efforts in this area.

\subsubsection{Commits Related to Major Logic Components Over Time}

\begin{figure}[ht]
    \centering
    \includegraphics[width=\linewidth]{feature-analysis/timeline_major_related_logic_component_smoothed.png}
    \caption{Commits Related to Major Logic Components Over Time}
    \label{fig:timeline_major_related_logic_component_smoothed}
\end{figure}

Figure~\ref{fig:timeline_major_related_logic_component_smoothed} illustrates the evolution of major logic components within the Linux eBPF subsystem over time. The \texttt{General Utilities} component, which includes test cases and build scripts, exhibits the most significant improvements, reaching a peak between 2022 and 2023 before experiencing a decline. In contrast, the \texttt{eBPF Instruction Logic} component displays two prominent peaks in 2018 and 2024, corresponding to the initial introduction of eBPF instructions and subsequent standardization efforts, respectively.

Other components, such as \texttt{Runtime Features, Helpers, and kfuncs}, show a notable peak in 2023 followed by a decrease and subsequent stabilization. Meanwhile, the \texttt{Control Plane Interface} and \texttt{Maps Logic} components maintain relatively steady levels of activity throughout the observed period.

\subsubsection{Use Cases or Events Over Time}

\begin{figure}[ht]
    \centering
    \includegraphics[width=\linewidth]{feature-analysis/timeline_usecases_or_submodule_events_smoothed.png}
    \caption{Use Cases or Events Over Time}
    \label{fig:timeline_usecases_or_submodule_events_smoothed}
\end{figure}

Figure~\ref{fig:timeline_usecases_or_submodule_events_smoothed} reveals significant fluctuations in event types over the years.

Notably, network-related events such as socket and XDP programs experienced a surge from 2020 to 2022, after which they entered a stabilization phase following the initial burst of feature additions and optimizations. The tracepoints-related activities show a steady increase with periodic fluctuations, peaking around 2021 before decreasing. The kprobe/ftrace-related events remain mostly stable, with a slight increase in 2022. Uprobe-related events show moderate activity over several years, with slight peaks in 2022 and again in 2024. The \texttt{struct\_ops}, an emerging feature introduced in 2020, shows a significant increase in activity between 2023 and 2024. Other events such as LSM for security remain minor.

\subsection{Deeper Insights Analysis}

This section delves into the survey responses to uncover patterns, trends, and areas for improvement within the eBPF subsystem.

\subsubsection{Which Kernel Components and Files Have the Most Frequent Bugs?}

\begin{figure}[ht]
    \centering
    \includegraphics[width=\linewidth]{feature-analysis/kernel_components_most_buggy_pie_chart.png}
    \caption{Kernel Implementation Components with the Most Bugs}
    \label{fig:buggy_kernel_component}
\end{figure}

Figure~\ref{fig:buggy_kernel_component} illustrates the kernel implementation components with a high number of bugs. While previous analyses and tools have primarily focused on improving the stability of the verifier and JIT compiler, these areas account for only about 35\% of the bugs. The largest number of bugs originate from eBPF event-related code, which involves the interaction of eBPF with other kernel subsystems. Additionally, helpers and maps also have a significant number of bugs. Due to the complexity of the control plane, the eBPF syscall interface is also prone to bugs.

By examining specific files, we observe that bugs frequently occur in the verifier, syscall, core, and network filter components. These files require better test coverage and more attention.

\textbf{Top 10 Buggy Files:}
\begin{verbatim}
kernel/bpf/verifier.c             425
net/core/filter.c                 140
kernel/bpf/syscall.c              111
include/linux/bpf.h                87
kernel/bpf/core.c                  83
include/uapi/linux/bpf.h           80
kernel/trace/bpf_trace.c           77
kernel/bpf/btf.c                   75
tools/include/uapi/linux/bpf.h     54
kernel/bpf/sockmap.c               51
\end{verbatim}

\subsubsection{What is the Relationship Between Instruction-Related Changes in the Verifier and All Verifier Bugs?}

\begin{figure}[ht]
    \centering
    \includegraphics[width=\linewidth]{feature-analysis/verifier_features_vs_general_bugs_over_time.png}
    \caption{Verifier Bugs or Features Related to eBPF Instructions Over Time}
    \label{fig:instruction_verifier_features_bugs_over_time}
\end{figure}

Figure~\ref{fig:instruction_verifier_features_bugs_over_time} shows that changes related to eBPF instructions in the verifier closely correlate with the number of verifier bugs. This insight highlights the importance of focusing on instruction-related aspects during verifier development and debugging to enhance overall system stability.

\subsubsection{The Evolution and Status of \texttt{libbpf}}

We also examined the lifecycle of specific components, such as \texttt{libbpf}.

Based on Figure~\ref{fig:libbpf_commit_classification}, the development of \texttt{libbpf} began to grow significantly in 2017. New feature development follows a cyclical pattern, with notable spikes around 2020 and 2022. After 2022, the number of cleanups and refactorings increased significantly, indicating a shift in focus towards code maintainability and stability. However, the decline in cleanup commits after 2023 suggests that while new features continue to be added, the emphasis has shifted more towards stabilization and optimization.

Historical milestones verify this trend. For instance, \texttt{libbpf} version 1.0 was released in August 2022~\cite{libbpf1}, and the major feature ``Compile Once, Run Everywhere'' (CO-RE) was introduced around 2020~\cite{core}, both aligning with the peaks and shifts observed in the commit history.

\begin{figure}[ht]
    \centering
    \includegraphics[width=\linewidth]{feature-analysis/libbpf_evolution_by_classification.png}
    \caption{Evolution of \texttt{libbpf} Over Time}
    \label{fig:libbpf_commit_classification}
\end{figure}

\subsubsection{What Dependencies Have Emerged Between Features and Components?}

Our analysis of feature-component dependencies in Figure~\ref{fig:feature_component_heatmap} reveals two primary patterns. First, new control plane abstractions such as \texttt{bpf\_link}\cite{bpflink} and \texttt{token}\cite{token} typically require coordinated updates to both the \texttt{syscall interface} and \texttt{libbpf}, indicating tightly coupled development. Second, runtime features like \texttt{bpf\_iter}\cite{bpf_iterators} and \texttt{spin\_lock}\cite{spinlock} mainly depend on internal kernel components such as \texttt{helpers} and \texttt{verifier}, with minimal impact on the \texttt{JIT compiler}, suggesting potential areas for JIT optimization.


\begin{figure}[ht]
    \centering
    \includegraphics[width=\linewidth]{feature-analysis/heatmap_bpf_keywords_vs_components.png}
    \caption{Feature-Component Interdependencies in the BPF Subsystem}
    \label{fig:feature_component_heatmap}
\end{figure}


\subsection{Survey Validation}

% To validate the effectiveness of \sys, we employed multiple approaches. First, we discussed the results with more than five eBPF experts who have submitted kernel patches or presented at conferences; they confirmed that the findings align with their understanding of eBPF evolution. We also shared the report with kernel maintainers and the BPF kernel mailing list for broader validation. Second, the survey responses proved to be generally clear and actionable, with minimal ambiguity in the structured data extracted from commits. Third, by comparing survey responses with real-world feature changes and historical milestones (such as the release of libbpf 1.0 and the introduction of CO-RE), we confirmed that the survey accurately captures the historical context of eBPF development. The consistency between our automated analysis and expert knowledge demonstrates that \sys can effectively transform unstructured development artifacts into reliable, structured insights about software evolution.
To validate the effectiveness of \sys, we employed multiple approaches. First, we discussed the multi-agent results with more than five eBPF experts who confirmed that the agent disagreements mirror real community tensions—maintainers indeed prioritize stability while contributors push features. We also shared the report with kernel maintainers and the BPF kernel mailing list, who validated that our virtual community accurately reflects real dynamics. Second, the multi-perspective responses revealed nuanced insights, with inter-agent disagreements highlighting genuine areas of community debate. Third, by comparing agent consensus patterns with real-world feature controversies (such as the BPF type format debates and verifier complexity discussions), we confirmed that \sys captures not just technical evolution but social dynamics. The correlation between agent disagreements and actual mailing list debates demonstrates that \sys effectively models how different stakeholders perceive kernel evolution, providing insights into collaboration patterns crucial for the FM era.

% \bibliographystyle{IEEEtran}
% \bibliography{references}
\section{Best Practices in the \emph{Code-Survey} Method}
\label{sec:best_practices}
The \emph{Code-Survey} methodology, particularly when integrated with Large Language Models (LLMs), requires carefully designed practices to ensure the accuracy and relevance of responses. The key principle is to create surveys intended for human participants and allow LLMs to perform on them; this approach leverages existing methodologies and can be easily adapted.

Below are some best practices to achieve reliable results:

\begin{enumerate}
    \item \textbf{Use Predefined Tags and Categories}: To minimize hallucinations or random answers from the LLM, it is essential to provide structured, closed-ended questions. Utilizing predefined tags such as ``Bug Fix,'' ``New Feature,'' or ``Performance Optimization'' helps standardize responses and reduces ambiguity.

    \item \textbf{Implement LLM Agent Workflows}: LLMs may need to review and refine their answers multiple times to improve accuracy. Incorporating feedback loops and techniques like ReAct~\cite{yao2022react} allows the model to re-evaluate its responses, enhancing overall data quality.

    \item \textbf{Allow for ``I'm Not Sure'' Responses}: Providing an option for the LLM to indicate uncertainty prevents random or misleading answers when encountering unfamiliar or complex questions. This is particularly useful in domains where the LLM's knowledge may be limited or incomplete.

    \item \textbf{Pilot Testing and Iterative Refinement}: Conducting pilot tests prior to full deployment helps identify potential issues with question clarity and LLM understanding. Iterative refinement of survey questions based on these trials ensures logical consistency and improves data reliability.

    \item \textbf{Ensure Consistency and Perform Data Validation}: Design questions for Consistency. After the survey, apply validation checks to ensure consistency across responses. Detecting and filtering out illogical or contradictory answers is essential for maintaining the integrity of the dataset.
\end{enumerate}

By adhering to these best practices, the \emph{Code-Survey} method can effectively leverage LLMs, mitigating risks such as hallucination and improving the reliability of responses in technical domains like Linux kernel commit analysis.

\section{Limitations}
\label{sec:limitations}

While \emph{Code-Survey} offers significant advancements in analyzing the evolution of large software systems, it has certain limitations:

\begin{enumerate}
    \item \textbf{Dependency on Data Quality}: The accuracy of \emph{Code-Survey} is heavily dependent on the quality and completeness of the input data. Incomplete commit messages, patches or fragmented email discussions can lead to gaps in the structured data, potentially obscuring important aspects of feature evolution.

    \item \textbf{Limitations of LLMs}: Although LLMs like GPT-4o are powerful, they are not infallible. Misinterpretations of commit messages or developer communications can result in inaccurate data structuring~\cite{ji2023survey}. LLMs may sometimes generate plausible but incorrect information (hallucinations) or miss important details in the questions~\cite{bubeck2023sparks}. To mitigate these issues, careful survey design and validation are essential to guide the model's responses more effectively.

    \item \textbf{Requirement for Human Expert Feedback}: Despite automation, human expertise remains essential for designing effective surveys, evaluating results, and ensuring the contextual relevance of the structured data. This dependency can limit the scalability of \emph{Code-Survey} in scenarios where expert availability is constrained.
\end{enumerate}

\section{Future Work}
\label{sec:future}

While \emph{Code-Survey} demonstrates significant potential in organizing and analyzing unstructured software data, several areas warrant further exploration and improvement:

\subsection{Enhanced Evaluation of LLM-Generated Survey Data}

To address challenges such as hallucinations and inaccuracies in LLM outputs, future work will focus on developing robust validation frameworks. This includes benchmarking results against curated datasets and involving human experts in refining LLM outputs. Enhancing the reliability of the structured data will improve the overall effectiveness of \emph{Code-Survey}.

\subsection{Performance Optimization with Advanced LLMs}

The current proof-of-concept showcases automation but with room for performance improvements if no human feedback is procided. Future efforts will explore the use of more advanced models, such as O1~\cite{o1}, and the implementation of multi-agent systems to optimize performance and accuracy. Ensuring compatibility with machine analysis tools is crucial for seamless integration into existing workflows.

\subsection{Application to Other Software Projects}

While \emph{Code-Survey} has been applied to the Linux eBPF subsystem, its methodology can be directly applied to other projects like Kubernetes, LLVM, and Apache. Expanding to these repositories will test its scalability and versatility, potentially requiring adjustments to accommodate different development practices and environments.

\subsection{Incorporation of Additional Data Sources like Code, Trace, and Execution Flow}

Due to the time limited, the case study currently mainly focus on analyzing the commit and features. Extending \emph{Code-Survey} to incorporate a wider range of data sources—such as source code, execution traces, and execution flows—will provide a more comprehensive understanding of software systems. Direct structuring of code and functions, transforming technical elements into structured, query-able data or graphs with attributes, will enable advanced analyses. This approach will facilitate a deeper exploration of software implementations, performance characteristics, and evolutionary patterns.

By pursuing these enhancements, \emph{Code-Survey} aims to become a comprehensive tool for analyzing complex software systems, benefiting both developers and researchers in the field of software engineering.

% \bibliographystyle{IEEEtran}
% \bibliography{references}

\section{Conclusion}

This paper introduced \emph{Code-Survey}, the first methodology that leverages LLMs for systematically exploring and analyzing large-scale codebase through a survey-based approach. Applied to the Linux eBPF subsystem, \emph{Code-Survey} successfully uncovered patterns in feature evolution and design that traditional methods overlook. Despite some limitations, our approach provides a valuable framework for understanding the growth of real-world software systems. Future work will expand its scope, enhance LLM capabilities, and apply \emph{Code-Survey} to other large-scale codebases. We also invite collaborators to work together on the ongoing development and refinement of this pioneering methodology.

% In this paper, we introduce \emph{Code-survey}, a novel approach that leverages LLMs to systematically transform unstructured data into structured datasets for analysis. By focusing on commit histories and developer communications, Code-survey enables us to answer questions that were previously impossible to tackle using only unstructured data in large real-world systems. Structured data analysis allows us to explore questions like:

% \begin{itemize}
%     \item \textbf{Design Rationale and Decision-Making:}
%     \begin{itemize}
%         \item What were the primary motivations behind introducing specific eBPF features or bug fixes?
%         \item How do design rationales discussed in developer communications correlate with implementation choices in commits?
%         \item What trade-offs have been considered in the design of eBPF features, such as flexibility vs. performance?
%     \end{itemize}
    
%     \item \textbf{Feature Evolution and Integration:}
%     \begin{itemize}
%         \item How has the functionality of a specific eBPF feature, like \texttt{bpf\_link}, evolved over successive commits?
%         \item What dependencies have emerged between eBPF features and other subsystems within the Linux kernel?
%         \item How do new feature introductions impact the stability and performance of existing eBPF features?
%     \end{itemize}
    
%     \item \textbf{Development Patterns and Trends:}
%     \begin{itemize}
%         \item What patterns can be observed in the frequency and nature of commits related to specific eBPF features over time?
%         \item Are there identifiable phases in the lifecycle of eBPF features, such as initial development, stabilization, and optimization?
%         \item How do periods of high commit activity correlate with major kernel releases or external events?
%     \end{itemize}
    
%     \item \textbf{Collaborative Dynamics and Communication:}
%     \begin{itemize}
%         \item How do discussions in mailing lists influence the direction and prioritization of eBPF feature development?
%         \item What roles do key maintainers play in shaping the evolution of the eBPF subsystem?
%         \item How does the collaboration between different contributors affect the consistency and coherence of eBPF feature implementations?
%     \end{itemize}
    
%     \item \textbf{Impact Assessment and Maintenance:}
%     \begin{itemize}
%         \item What are the common causes of feature regressions in eBPF, and how are they addressed in subsequent commits?
%         \item How do maintainers assess the long-term maintenance needs of eBPF features based on commit history and developer feedback?
%         \item What metrics can be derived from structured data to evaluate the reliability and performance improvements of eBPF features?
%     \end{itemize}
    
%     \item \textbf{Adoption and Usage Insights:}
%     \begin{itemize}
%         \item How has the adoption of eBPF features like \texttt{bpf\_link} grown within the Linux kernel, and what factors have driven this adoption?
%         \item What usage patterns emerge from the commit history that indicate how end-users interact with specific eBPF features?
%         \item How do enhancements to eBPF influence its applicability in emerging domains such as cloud-native environments and security monitoring?
%     \end{itemize}
    
%     \item \textbf{Knowledge Transfer and Documentation:}
%     \begin{itemize}
%         \item How effectively do commit messages and mailing list discussions convey the necessary information for future maintenance and development?
%         \item What gaps exist between developer communications and the actual codebase, and how can structured data help bridge these gaps?
%         \item How does the clarity and detail of commit messages impact the ease of understanding feature evolution for new contributors?
%     \end{itemize}
    
%     \item \textbf{Comparative Analysis Across Subsystems:}
%     \begin{itemize}
%         \item How does the evolution of eBPF compare to other subsystems within the Linux kernel in terms of complexity and development pace?
%         \item What lessons can be learned from the development history of eBPF that can be applied to improving other kernel subsystems?
%         \item Are there common factors that contribute to the successful integration and maintenance of features across different kernel subsystems?
%     \end{itemize}
    
%     \item \textbf{Technical Design and Implementation Components:}
%     \begin{itemize}
%         \item How do logical design principles of eBPF translate into specific implementation components within the kernel?
%         \item What are the key implementation challenges faced during the development of eBPF, and how were they overcome?
%         \item How have optimizations in eBPF’s implementation impacted its overall performance and efficiency?
%         \item How does the implementation of eBPF ensure compatibility and interoperability with other kernel subsystems?
%         \item What role do helper functions play in the implementation of eBPF features, and how have they evolved over time?
%         \item How does the implementation of the eBPF verifier contribute to the security and reliability of eBPF programs?
%     \end{itemize}
    
%     \item \textbf{Logic Design and Real Components:}
%     \begin{itemize}
%         \item How does the logical design of eBPF programs influence their implementation within the Linux kernel?
%         \item What are the possible logical components and their corresponding implementation components in eBPF?
%         \item How do design criteria of the eBPF verifier affect the implementation of eBPF helper functions?
%         \item How have the interactions between eBPF’s logical components and other kernel subsystems been managed in the implementation?
%         \item How do logical design changes in eBPF correlate with implementation modifications in the kernel codebase?
%     \end{itemize}
    
%     \item \textbf{Deep Insight Questions for Design and Implementation Gap:}
%     \begin{itemize}
%         \item How do logical abstractions in eBPF design ensure seamless integration with kernel-level operations?
%         \item What are the key design principles that guided the implementation of persistent eBPF program attachments like \texttt{bpf\_link}?
%         \item How has the design of eBPF’s execution environment influenced its implementation efficiency and security?
%         \item



% In this paper, we introduce \emph{Code-survey}, a novel approach that leverages LLMs to systematically transform unstructured data into structured datasets for analysis. By focusing on commit histories and developer communications, Code-survey enables us to answer questions that were previously impossible to tackle using only unstructured data in large real-world systems. Structured data analysis allows us to explore questions like:





 

% We make several interesting observations: 

% - Call attention to under-explored areas of the eBPF tool chain, runtime to support the various use cases. 

% - Highlight interesting research directions. 

% We also deployed a open-source pipeline and database for real time open-source eBPF community, which can help you check and analysis quantatively, and can be easily applied to other Linux subsystems.											 

 

% ## the challendges of Analysis Linux Community: 

 

% Too much data in one of the largest community of the world 

% Unstruct data, including reviews, patchs, annoounces, discussions all together 

% Self host on mail without platforms like github 

 

% It’s hard to use traditional NLP mothod or data metrics to analysis them. Doing survey with kernel experts need a huge amond of money and 

 

% ## Methods 

 

 

% Kernel or eBPF Experts give a Survey and Questionnaire 

% Collect meta data of commits and mails with traditional method, including timestamps, mail, thread logic information 

% Connect the data 

 

% Allow llm to answer the Survey and Questionnaire based on the mails and patches,  

% ## Validate the results with kernel experts 

% Talk to them and collect feed backs. Also evalute the performance of the system. 

% eBPF the goal is to create innovation. The eBPF is continue envlove and no standard. 

% - Tools and verifier need to be portable
% - 

\bibliographystyle{plain}
\bibliography{cite}
\end{document}
