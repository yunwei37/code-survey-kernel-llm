
\section{Design}
\label{sec:design}

\sys creates a high-fidelity simulation of the open-source community's collaborative structure. Unlike existing multi-agent systems that rely on generic roles, \sys instantiates \textbf{Empirically-Grounded Personas}: agents representing actual key developers to capture the nuanced perspectives that shape kernel evolution. The framework employs carefully designed surveys that agents answer based on development artifacts.

\subsection{Design Goals}

Three core principles guide \sys's design. First, \textbf{Faithful Stakeholder Modeling} (G-1) requires agents to embody distinct perspectives of real kernel participants, mirroring tensions between feature velocity, long-term stability, and risk mitigation. Second, \textbf{Question-Driven, Evidence-Based Analytics} (G-2) mandates that the system answer high-level natural language questions with quantitative analysis and \textbf{explicit, traceable evidence}, reflecting the kernel community's norm where any claim must be backed by references to specific commits, mailing list threads, or performance data. Finally, \textbf{Measurable Reliability and Consensus} (G-3) requires the system to quantify its own internal agreement and reliability by measuring inter-agent agreement, building a trust model analogous to the kernel's \texttt{Reviewed-by} and \texttt{Tested-by} tags~\cite{kerneldocs-submitting}, ensuring that outputs are the result of a \textbf{consensus between key developer perspectives}.


\subsection{Simulating the Kernel Development Workflow}

The \sys architecture simulates the kernel patch lifecycle through five phases. To illustrate, consider how \sys would answer the query: \textbf{"How did the refactoring of verifier loops after kernel v5.10 affect BPF program size limits?"}

\emph{Figure X: The architecture of \sys, which simulates the kernel development workflow to analyze historical artifacts.}

\begin{enumerate}
\item \textbf{Phase 1: Query Decomposition \& Artifact Retrieval.} The \textbf{Orchestrator} agent parses the user's natural language query, identifying core topics (verifier, performance, limits) and the relevant timeframe (post-v5.10). It selects a tailored survey schema for architectural changes, including questions about motivation, technical risk, performance impact, and security implications. The agent then retrieves relevant historical artifacts: commits, patch series, and mailing list threads related to verifier loop refactoring.

\item \textbf{Phase 2: Automated Triage.} The retrieved artifacts pass through a \textbf{Triage Crew} simulating the kernel's automated CI systems. \texttt{SyzbotTwin} flags any artifacts related to fuzzer-found bugs in the verifier loops~\cite{vyukov2020syzkaller}, while \texttt{CheckpatchBot} annotates for style compliance~\cite{kerneldocs-checkpatch}. This phase enriches the artifacts with objective metadata before expert review.

\item \textbf{Phase 3: Asynchronous Community Review (Mailing List Simulation).} The core analytical phase begins as a dynamically assembled \textbf{Persona Crew} analyzes the artifacts in parallel. The \textbf{Alexei Starovoitov agent}, drawing from its maintainer memory, completes its survey noting the motivation was "refactoring for performance" with a "medium" technical risk. The \textbf{Daniel Borkmann agent} concurs on the motivation but assesses the risk as "low," citing extensive test suite development. The \textbf{Brendan Gregg agent} highlights that the primary "impact" was enabling more powerful tracing tools. Each agent records findings in a structured survey, transforming qualitative perspectives into analyzable data.

\item \textbf{Phase 4: Consensus Building \& Maintainer Decision (Resolution Simulation).} The \textbf{Critic-Resolver} agent aggregates surveys and calculates inter-agent agreement (Cohen's kappa, $\kappa$), flagging the disagreement on risk assessment. This triggers a simulated debate: the Alexei agent cites a past verifier bug, while the Daniel agent provides links to specific test-suite emails. They reconcile to a consensus of "low-to-medium risk, mitigated by extensive testing," analogous to a maintainer applying a \texttt{Reviewed-by} tag~\cite{kerneldocs-submitting}.

\item \textbf{Phase 5: Insight Synthesis \& Reporting.} The \textbf{Data-Synthesizer} agent performs quantitative analysis on the structured survey data, plotting the \texttt{BPF\_MAX\_INSNS} constant across kernel versions to visually confirm the impact on program size limits. It composes a final report integrating the narrative explanations from agent surveys with quantitative charts, delivering evidence-backed insights to answer the original query.
\end{enumerate}


\subsection{Empirically-Grounded Personas}

Central to \sys's fidelity is the deployment of Empirically-Grounded Personas: computational agents that embody the expertise and perspectives of actual kernel contributors. Each persona's analytical framework derives from the developer's historical contributions, replicating their domain-specific insights and decision patterns. Each persona is underpinned by a comprehensive Knowledge Base, implemented as a dedicated vector store for Retrieval-Augmented Generation (RAG), which encompasses: (i) all commits authored by the developer, (ii) their complete mailing list correspondence, (iii) patches formally reviewed (identified through \texttt{Reviewed-by:} tags~\cite{kerneldocs-submitting}), and (iv) documented positions from design discussions and technical presentations.

This architecture enables highly specialized, perspective-driven analysis that mirrors the heterogeneous expertise within the kernel community. For instance, the \textbf{Alexei Starovoitov agent}, representing the eBPF maintainer, evaluates changes through the lens of verifier safety and API stability, leveraging its repository of thousands of accepted and rejected BPF patches. Conversely, the \textbf{Brendan Gregg agent}, embodying observability expertise, assesses the same changes based on their utility for performance tracing tools, drawing upon extensive knowledge of \texttt{perf} and \texttt{bpftrace} evolution. The \textbf{KP Singh agent} applies a security-focused perspective, analyzing code patterns against a corpus of Linux Security Module (LSM) discussions and BPF hardening initiatives. Through this multi-perspective approach, \sys enables systematic socio-technical analysis by leveraging the diverse viewpoints encoded in developer histories.

